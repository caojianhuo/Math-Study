\documentclass{ctexart}
\usepackage{fancyhdr}
\usepackage{amsmath}
\usepackage{indentfirst}
\usepackage{graphicx}
\pagestyle{fancy}
\lfoot{}%这条语句可以让页码出现在下方
\title{《抽象代数》\\ 第二次作业} 
\author{\\姓名:姜岚曦 \ \ 学号:19375233
		\\姓名:魏来   \ \ 学号:20374104         
	    \\姓名:曹建钬 \ \ 学号:20375177
        \\姓名:李璞   \ \ 学号:20376164
        \\姓名:刘炅   \ \ 学号:21374261}
\date{}
\begin{document}
\pagestyle{empty}
\thispagestyle{empty}
\CTEXsetup[format={\Large\bfseries}]{section}
\maketitle
\clearpage	
	
\section*{\$1.7:一一映射、变换}
1. $A = \{$所有$>0$的数$\}$,$\bar{A}$ = $\{$所有实数$\}$. 找一个$A$与$\bar{A}$间的一一映射.\\
解: \\
定义 $\phi: \ \ \ \ \ a \leftrightarrow \lg{a}$
是满足要求的一一映射 \\
对于$\forall a \in A, \exists \bar{a} = \lg a,$且$\bar{a} \in \bar{A}$. \\
对于$\forall \bar{b} \in \bar{A}, \exists b = e^{\bar{b}} \in A$.\\
故$\phi$既为单射,亦为满射,即一一映射.

2. $A = \{$所有$\geq0$的数$\}$,$\bar{A}$ = $\{$所有实数$\bar{a}, 0 \leq \bar{a} \leq 1\}$. 找一个$A$到$\bar{A}$的满射. \\
解:\\
$\phi: \ \ \ \ \ a \rightarrow |\sin a|$ \\
对于$\forall \bar{a} \in \bar{A}, \exists a = |\arcsin \bar{a}| \in A.$ \\
故$\phi$是一个$A$到$\bar{A}$的满射.

3. 假定$\phi$是$A$与$\bar{A}$间的一个一一映射,$a$是$A$的一个元.
\begin{equation*}
	\phi^{-1}[\phi(a)] = ? \ \ \phi[\phi^{-1}(a)] = ?
\end{equation*}
若$\phi$是$A$的一个一一变换,这两个问题的回答又该是什么? \\
解:\\
\begin{itemize}
	\item 若$\phi$是$A$与$\bar{A}$间的一个一一映射,$\phi^{-1}[\phi(a)] = a$,但$a$不一定是$\bar{A}$中的元素,$\phi[\phi^{-1}(a)]$未必有意义,除非$\bar{A}=A$.
	\item 若$\phi$是$A$的一个一一变换,则$	\phi^{-1}[\phi(a)] = a $,亦有$\phi[\phi^{-1}(a)] = a$
\end{itemize}

\section*{\$1.8:同态}
1. $A=\{ $所有实数$x\}$. $A$的代数运算是普通乘法. 以下映射是不是$A$到$A$的一个子集$\bar{A}$的同态满射? 
\begin{equation*}
	a)\  x \rightarrow |x| \ \ 
	b)\  x \rightarrow 2x  \ \ 
	c)\  x \rightarrow x^2 \ \ 
	d)\  x \rightarrow -x  \ \ 
\end{equation*}
解:\\ 
\begin{itemize}
	\item[$a)$] $x \rightarrow |x|$ \\
	首先对$\forall \bar{a} = |a| \in \bar{A}, \exists a = -\bar{a}$或$\bar{a}\in A$. 即$a)$是一个满射. \\
	对于$\forall a,b \in A, \bar{a} = |a|, \bar{b} = |b|$.\\ $a \circ b = ab, \bar{a} \circ \bar{b} = \bar{a}\bar{b} = |a| \cdot |b| = |ab| = \phi(a \circ b)$. 故有$a \circ b \rightarrow \bar{a} \circ \bar{b}$ \\
	故$a)$是同态满射
	\item[$b)$] $x \rightarrow 2x$ \\
	显然$b)$也是一个满射. \\
	对于$\forall a,b \in A, \bar{a} = 2a, \bar{b} = 2b$.\\ $a \circ b = ab, \bar{a} \circ \bar{b} = 2a \cdot 2b = 4ab \neq \phi(a \circ b)=2ab$. 式中不等号在$ab \neq 0$时始终成立. \\
	故$b)$不是同态满射	
	\item[$c)$] $x \rightarrow x^2$ \\
	显然对$\forall \bar{a} \in \bar{A}, \exists a = \sqrt{\bar{a}} \in A$. 即$c)$是一个满射. \\
	对于$\forall a,b \in A, \bar{a} = a^2, \bar{b} = b^2$.\\ $a \circ b = ab, \bar{a} \circ \bar{b} = a^2b^2 = \phi(a \circ b) = a^2b^2$. \\
	故$a)$是同态满射
	\item[$d)$] $x \rightarrow -x$  \\
	显然$d)$是一个满射. \\
	对于$\forall a,b \in A, \bar{a} = -a, \bar{b} = -b$.\\ $a \circ b = ab, \bar{a} \circ \bar{b} = ab \neq \phi(a \circ b) = -ab$.在$ab \neq 0$时成立. \\
	故$d)$不是同态满射
\end{itemize}

2. 假定$A$和$\bar{A}$对于代数运算$\circ$和$\bar{\circ}$来说同态,$\bar{A}$和$\bar{\bar{A}}$对于代数运算$\bar{\circ}$和$\bar{\bar{\circ}}$来说同态. 证明,$A$和$\bar{\bar{A}}$对于代数运算$\circ$和$\bar{\bar{\circ}}$来说同态. \\
证明:\\
对于$\forall a \in A, \exists \phi_1: \ \ \ \ a \rightarrow \bar{a}, \bar{a} \in \bar{A}.$ \\
对于$\forall \bar{b} \in \bar{A}, \exists \phi_2: \ \ \ \ \bar{b} \rightarrow \bar{\bar{b}}, \bar{\bar{b}} \in \bar{\bar{A}}.$ \\
且有 $\forall a,b \in A$,有$\phi_1(a \circ b) = \phi_1 a \ \bar{\circ} \  \phi_1 b$\\
 $\forall c,d \in \bar{A}$,有$\phi_2(c \ \bar{\circ} \ d) = \phi_2 c \  \bar{\bar{\circ}} \ \phi_2 d$. \\
令$c = \phi_1 a , d = \phi_1 b$,有$\phi_1 (a \circ b) = c \bar{\circ} d$ \\
即有$\phi_2(\phi_1(a \circ b)) = \phi_2 (\phi_1 a) \bar{\bar{\circ}} \phi_2(\phi_1 b)$ \\
定义从$A$到$\bar{\bar{A}}$的映射$\phi_3: \ \ a \rightarrow \phi_2(\phi_1(a))$. \\
则可知$A$和$\bar{\bar{A}}$对于$\circ$和$\bar{\bar{\circ}}$同态.
\section*{\$1.9:同构、自同构}
1. $A=\{a, b, c\}$. 代数运算$\circ$由下表给定 \\
\begin{center}
	\begin{tabular}{c|ccc}
		& a & b & c \\
		\hline
		a & c & c & c \\
		b & c & c & c \\
		c & c & c & c \\
	\end{tabular} 
\end{center}
找出所有$A$的一一变换,对于代数运算$\circ$来说,这些一一变换是否都是$A$的自同构?\\
解:\\
\begin{equation*}
	\phi_1: \ a \rightarrow b, \ b \rightarrow c, \ c \rightarrow a
\end{equation*}
\begin{equation*}
	\phi_2: \ a \rightarrow a, \ b \rightarrow b, \ c \rightarrow c
\end{equation*}
\begin{equation*}
	\phi_3: \ a \rightarrow a, \ b \rightarrow c, \ c \rightarrow b
\end{equation*}
\begin{equation*}
	\phi_4: \ a \rightarrow b, \ b \rightarrow a, \ c \rightarrow c
\end{equation*}
\begin{equation*}
	\phi_5: \ a \rightarrow c, \ b \rightarrow a, \ c \rightarrow b
\end{equation*}
\begin{equation*}
	\phi_6: \ a \rightarrow c, \ b \rightarrow b, \ c \rightarrow a
\end{equation*}
$A$的任意一一变换对于$\circ$而言均是$A$的自同构.

2. $A=\{$所有有理数$\}$. 找一个$A$的对于普通加法来说的自同构(映射$x \leftrightarrow x$除外). \\
解:\\
设$\{ \phi_k \}$是$A \rightarrow A$的一系列映射. \\
考虑映射$\phi_k : \ \  a \rightarrow \bar{a} = ka$,其中$k$是任意不等于1的有理数. \\
对于$\forall a,b \in A, a \circ b = a + b, \bar{a} = ka, \bar{b} = kb.$\\
$\phi_k(a \circ b) = k(a+b) = \phi_k(a) \circ \phi_k(b) = ka+kb$ \\
显然,任何形如$\phi_k$的映射均是一一映射,故都是$A$的自同构.

3*.\\
解:\\
假设存在一个一一映射$\phi: \ \ A \rightarrow \bar{A}$. 使得其为$A$到$\bar{A}$的同构映射. \\
令$a \in A, \phi(a)=\bar{a} \in \bar{A}$. \\
由同构映射,有$\phi(\frac{a}{2}+\frac{a}{2})=\phi(\frac{a}{2})\cdot \phi(\frac{a}{2})$成立. \\
即$\phi(a)=\phi^2(\frac{a}{2})$ \\
由于$\bar{A}$是有理数集的子集,故$\phi^2(\frac{a}{2})>0$,即$\phi(a)>0$对任意$a \in A$成立. \\
意味着任意$\bar{b}<0$,且$\bar{b} \in \bar{A}$将找不到$\phi^{-1}$对应的$A$中原象,与$\phi$是一一映射矛盾,放$\phi$不存在.
\section*{\$1.10:等价关系与集合的分类}
1. $A=\{$所有实数$\}$. $A$的元间的关系$>$以及$\geq$是不是等价关系? \\
解:\\
对于关系$>$,显然不符合反射律,$ \forall a \in A, $不存在$ a>a $.故非等价关系
\\
对于关系$\geq$,符合反射律:$ \forall a \in A, $有$a \geq a$成立. 
不符合对称律:$\forall a,b \in A$且$a \neq b$,$a \geq b$与$b \geq a$只能成立其一.\\
故非等价关系.\\

2*. \\
解:\\
即便满足对称律与推移律.对于$a$而言,假如不存在一个使得$aRb$成立的$b$,就无法得到$aRa$.\\
反由对称律与推移律的存在无法确保如是的$b$存在.

3. 仿照例3规定整数间的关系:
\begin{equation*}
	a \equiv b (-5)
\end{equation*}
证明你所规定的是一个等价关系,并且找出模$-5$的剩余类.\\
证明: \\
反射律:$a \equiv a(-5)$显然成立. \\
对称律:$a \equiv b(-5) \Rightarrow b \equiv a(-5)$显然成立. \\
推移律:$a \equiv b(-5), b \equiv c(-5) \Rightarrow a \equiv c(-5)$,显然成应.\\
定义$-5$的剩余类$\{0,1,2,3,4\}$
\end{document}