\documentclass{ctexart}
\usepackage{fancyhdr}
\usepackage{amsmath}
\usepackage{indentfirst}
\usepackage{graphicx}
\pagestyle{fancy}
\lfoot{}%这条语句可以让页码出现在下方
\title{《抽象代数》\\ 第一次作业} 
\author{\\姓名:姜岚曦 \ \ 学号:19375233
		\\姓名:魏来   \ \ 学号:20374104         
	    \\姓名:曹建钬 \ \ 学号:20375177
        \\姓名:李璞   \ \ 学号:20376164
        \\姓名:刘炅   \ \ 学号:21374261}
\date{}
\begin{document}
\pagestyle{empty}
\thispagestyle{empty}
\CTEXsetup[format={\Large\bfseries}]{section}
\maketitle
\clearpage	
	
\section*{\$1.2:映射}
1. $A = \{1,2,3, \cdots, 100\}$. 找一个$A × A$到$A$的映射. \\
解: \\
定义 $\phi: \ \ \ \ \ (a_1,a_2) \longrightarrow \lfloor\frac{1}{2}(a_1+a_2)\rfloor=\phi(a_1,a_2)$
为$A×A$到$A$的一个映射 \\
其中$\lfloor\frac{1}{2}(a_1+a_2)\rfloor$表示不超过$\frac{1}{2}(a_1+a_2)$的最大整数

2. 在你为习题$1$所找到的映射之下,是不是$A$的每一个元都是$A×A$的一个元的象? \\
解:\\
是,对于$\forall x \in A, \exists (a_1,a_2)=(x,x), s.t. \phi(x,x)=x$
\section*{\$1.3:代数运算}
1. $A=\{ $所有不等于零的偶数$\}$. 找一个集合$D$,使得普通除法是$A×A$到$D$的代数运算,是不是找得到一个以上的这样的$D$? \\
解:\\ 可以找到一个以上这样的$D$ \\
令$D_0=Q$,即有理数的全体构成的集合 \\
那么包含$D_0$的任意集合都是符合要求的$D$,故$D$肯定不止一个

2. $A= \{a,b,c\}$. 规定$A$的两个不同的代数运算. \\
解:\\
$R_1$和$R_2$都是$A= \{a,b,c\}$的代数运算:\\
$R_1:$\begin{center}
	\begin{tabular}{c|ccc}
		& a & b & c \\
		\hline
		a & a & b & c \\
		b & a & b & c \\
		c & a & b & c \\
	\end{tabular} 
\end{center}
$R_2:$\begin{center}
	\begin{tabular}{c|ccc}
		& a & b & c \\
		\hline
		a & b & b & b \\
		b & b & b & b \\
		c & b & b & b \\
	\end{tabular} 
\end{center}
\section*{\$1.4:结合律}
1. $A=\{$所有不等于零的实数$\}$. $\circ$是普通除法:$a \circ b = \frac{a}{b}$. 这个代数运算适合不适合结合律? \\
解:\\
考虑:
\begin{equation*}
	(a \circ b)\circ c = \frac{a}{b} \circ c = \frac{a}{bc} 
\end{equation*}
\begin{equation*}
	a \circ (b \circ c) = a \circ \frac{b}{c} = \frac{ac}{b} 
\end{equation*}
由于$a,b,c \in A=\{x | x \in R,x \neq 0\}$任意性,一般没有$\frac{a}{bc}=\frac{ac}{b}$,即
$(a \circ b)\circ c=a \circ (b \circ c)$未必成立,故代数运算$\circ$不适合结合律 

2. $A=\{$所有实数$\}$. \\
\begin{equation*}
	\circ: \ \ \ \ \ \ (a,b) \rightarrow a+2b = a \circ b
\end{equation*}
\noindent
这个代数运算适合不适合结合律?\\
解:\\
\begin{equation*}
	(a \circ b)\circ c = (a+2b) \circ c = a+2b+2c 
\end{equation*}
\begin{equation*}
	a \circ (b \circ c) = a \circ (b+2c) = a+2b+4c 
\end{equation*}
由于$a,b,c \in A=\{$所有实数$\}$任意性,代数运算$\circ$不适合结合律

3. $A=\{a,b,c\}$. 由表:\\
\begin{center}
\begin{tabular}{c|ccc}
	  & a & b & c \\
	\hline
	a & a & b & c \\
	b & b & c & a \\
	c & c & a & b \\
\end{tabular} 
\end{center}
所给的代数运算适合不适合结合律?\\
解: \\
\begin{equation*}
	a \circ b \circ c = a 
\end{equation*}
\begin{equation*}
	a \circ (b \circ c) = a \circ a = a
\end{equation*}
依次替换$3^3=27$次进行验证可证得此代数运算适合结合律
\section*{\$1.5:交换律}
1. $A=\{$所有实数$\}$. $\circ$是普通减法:$a \circ b = a-b$. 这个代数运算适合不适合交换律? \\
解:\\
\begin{equation*}
	a \circ b  = a - b
\end{equation*}
\begin{equation*}
	b \circ a = b - a
\end{equation*}
显然,此代数运算不适合交换律

2. $A=\{a,b,c,d\}$. 由表:\\
\begin{center}
	\begin{tabular}{c|cccc}
		& a & b & c & d\\
		\hline
		a & a & b & c & d\\
		b & b & d & a & c\\
		c & c & a & b & d\\
		d & d & c & a & b\\
	\end{tabular} 
\end{center}
所给的代数运算适合不适合交换律?\\
解: \\
若代数运算是符合交换律的,则其运算应为对称阵\\
可见 $c \circ d = d \neq a = d \circ c$,故不适合交换律

\section*{\$1.6:分配率}
1. 假定$\bigodot$,$\bigoplus$是$A$的两个代数运算,并且$\bigoplus$适合结合律,$\bigodot$,$\bigoplus$适合两个分配率. 证明
\begin{align*}
	 (a_1 \odot b_1)\oplus(a_1 \odot b_2)\oplus(a_2 \odot b_1)\oplus(a_2 \odot b_2) \\
	= (a_1 \odot b_1)\oplus(a_2 \odot b_1)\oplus(a_1 \odot b_2)\oplus(a_2 \odot b_2)
\end{align*}
证明:
\begin{align*}
	(a_1 \odot b_1)\oplus(a_1 \odot b_2)\oplus(a_2 \odot b_1)\oplus(a_2 \odot b_2) \\
	= (a_1 \odot (b_1 \oplus b_2)) \oplus (a_2 \odot (b_1 \oplus b_2)) \\
	= (a_1 \oplus a_2) \odot (b_1 \oplus b_2)
\end{align*}
\begin{align*}
	(a_1 \odot b_1)\oplus(a_2 \odot b_1)\oplus(a_1 \odot b_2)\oplus(a_2 \odot b_2) \\
	= ((a_1 \oplus a_2) \odot b_1) \oplus ((a_1 \oplus a_2) \odot b_2) \\
	= (a_1 \oplus a_2) \odot (b_1 \oplus b_2)
\end{align*}
两式相等,原命题得证。
\end{document}