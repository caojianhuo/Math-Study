\documentclass{ctexart}
\usepackage{fancyhdr}
\usepackage{amsmath}
\usepackage{indentfirst}
\usepackage{graphicx}
\pagestyle{fancy}
\lfoot{}%这条语句可以让页码出现在下方
\title{《抽象代数》\\ 第三、四次作业} 
\author{\\姓名:姜岚曦 \ \ 学号:19375233
		\\姓名:魏来   \ \ 学号:20374104         
	    \\姓名:曹建钬 \ \ 学号:20375177
        \\姓名:李璞   \ \ 学号:20376164
        \\姓名:刘炅   \ \ 学号:21374261}
\date{}
\begin{document}
\pagestyle{empty}
\thispagestyle{empty}
\CTEXsetup[format={\Large\bfseries}]{section}
\maketitle
\clearpage	
	
\section*{\$2.1:群的定义}
1. 
解: \\
$A = \{$全体整数$\}$\\
I.两个整数相减还是一个整数 \\
II.$a-(b-c)=a-b+c\neq (a-b)-c$\\
故不满足结合律,$A$不是一个群

2. 
解:\\
$A=\{0,1\}$,定义运算$R: 01=1\ \ \ \ 10=1 \ \ \ \ 00=0 \ \ \ \ 11=0$\\
I.$A$对$R$而言是闭的\\
II.\\ $0(00)=00=0=00=(00)0 \ \ \ \ 1(00)=10=1=10=(10)0$ \\
$0(01)=01=1=01=(00)1 \ \ \ \ 1(01)=11=0=11=(10)1$ \\
$0(10)=01=1=10=(01)0 \ \ \ \ 1(10)=11=0=00=(11)0$ \\
$0(11)=00=0=11=(01)1 \ \ \ \ 1(11)=10=1=01=(11)1$ \\
说明结合律成立\\
III.\\
$1x=0 \Rightarrow x=1 \ \ \ \ y0=0 \Rightarrow y=0$\\
$0x=1 \Rightarrow x=1 \ \ \ \ y0=1 \Rightarrow y=1$\\
$0x=0 \Rightarrow x=0 \ \ \ \ y1=0 \Rightarrow y=1$\\
$1x=1 \Rightarrow x=0 \ \ \ \ y1=1 \Rightarrow y=0$\\
说明方程$ax=b,ya=b$都在$A$中有解\\
故$A$符合群的第一定义,是一个群

3. 
解:\\
由群的定义III,方程$ax=a$存在群$A$内的一解设为$e$,即$ae=a$\\
对于$\forall b \in A,ya=b$.存在一解$c$,即$ca=b$.\\
则对于$\forall b \in A$,有$be=cae=c(ae)=ca=b$成立.故IV'成立\\
由于$e\in A$,故$ax=e$有解.故V'成立\\
$a=ae=a(a^{-1}a)=(aa^{-1})a=ea$对$\forall a\in A$成立. \\
对于方程$ax=b$,取$x=a^{-1}b$,有$a(a^{-1}b)=eb =b$成立.\\
对于方程$ya=b$,取$y=ba^{-1}$,有$(ba)^{-1}a=b(a^{-1}a)=be=b$成立.\\
故IV'与V'和III等价,即I、II、IV'、V'可作为群的定义.

\section*{\$2.2:单位元,逆元,消去律}
1*. 
解:\\ 
若对VxeG,有xie成立
根据定理2,有 x=X对VXGG度立由题设有(ab)(ab)=e 即成立另面,有(ba)(ab)= b(aa)b = beb =hb=C感故(ab)(ab)=(ba)(ab)
等式两边在乘(ab)有(ab)(ab)(ab)7=(ba)(ab)(ab)即pabe=bae得到a=ba对Va;beC成立.救G是交换群证牛

2. 假定$A$和$\bar{A}$对于代数运算$\circ$和$\bar{\circ}$来说同态,$\bar{A}$和$\bar{\bar{A}}$对于代数运算$\bar{\circ}$和$\bar{\bar{\circ}}$来说同态. 证明,$A$和$\bar{\bar{A}}$对于代数运算$\circ$和$\bar{\bar{\circ}}$来说同态. \\
证明:\\
对于$\forall a \in A, \exists \phi_1: \ \ \ \ a \rightarrow \bar{a}, \bar{a} \in \bar{A}.$ \\
对于$\forall \bar{b} \in \bar{A}, \exists \phi_2: \ \ \ \ \bar{b} \rightarrow \bar{\bar{b}}, \bar{\bar{b}} \in \bar{\bar{A}}.$ \\
且有 $\forall a,b \in A$,有$\phi_1(a \circ b) = \phi_1 a \ \bar{\circ} \  \phi_1 b$\\
 $\forall c,d \in \bar{A}$,有$\phi_2(c \ \bar{\circ} \ d) = \phi_2 c \  \bar{\bar{\circ}} \ \phi_2 d$. \\
令$c = \phi_1 a , d = \phi_1 b$,有$\phi_1 (a \circ b) = c \bar{\circ} d$ \\
即有$\phi_2(\phi_1(a \circ b)) = \phi_2 (\phi_1 a) \bar{\bar{\circ}} \phi_2(\phi_1 b)$ \\
定义从$A$到$\bar{\bar{A}}$的映射$\phi_3: \ \ a \rightarrow \phi_2(\phi_1(a))$. \\
则可知$A$和$\bar{\bar{A}}$对于$\circ$和$\bar{\bar{\circ}}$同态.
\section*{\$1.9:同构、自同构}
1. $A=\{a, b, c\}$. 代数运算$\circ$由下表给定 \\
\begin{center}
	\begin{tabular}{c|ccc}
		& a & b & c \\
		\hline
		a & c & c & c \\
		b & c & c & c \\
		c & c & c & c \\
	\end{tabular} 
\end{center}
找出所有$A$的一一变换,对于代数运算$\circ$来说,这些一一变换是否都是$A$的自同构?\\
解:\\
\begin{equation*}
	\phi_1: \ a \rightarrow b, \ b \rightarrow c, \ c \rightarrow a
\end{equation*}
\begin{equation*}
	\phi_2: \ a \rightarrow a, \ b \rightarrow b, \ c \rightarrow c
\end{equation*}
\begin{equation*}
	\phi_3: \ a \rightarrow a, \ b \rightarrow c, \ c \rightarrow b
\end{equation*}
\begin{equation*}
	\phi_4: \ a \rightarrow b, \ b \rightarrow a, \ c \rightarrow c
\end{equation*}
\begin{equation*}
	\phi_5: \ a \rightarrow c, \ b \rightarrow a, \ c \rightarrow b
\end{equation*}
\begin{equation*}
	\phi_6: \ a \rightarrow c, \ b \rightarrow b, \ c \rightarrow a
\end{equation*}
$A$的任意一一变换对于$\circ$而言均是$A$的自同构.

2. $A=\{$所有有理数$\}$. 找一个$A$的对于普通加法来说的自同构(映射$x \leftrightarrow x$除外). \\
解:\\
设$\{ \phi_k \}$是$A \rightarrow A$的一系列映射. \\
考虑映射$\phi_k : \ \  a \rightarrow \bar{a} = ka$,其中$k$是任意不等于1的有理数. \\
对于$\forall a,b \in A, a \circ b = a + b, \bar{a} = ka, \bar{b} = kb.$\\
$\phi_k(a \circ b) = k(a+b) = \phi_k(a) \circ \phi_k(b) = ka+kb$ \\
显然,任何形如$\phi_k$的映射均是一一映射,故都是$A$的自同构.

3*.\\
解:\\
假设存在一个一一映射$\phi: \ \ A \rightarrow \bar{A}$. 使得其为$A$到$\bar{A}$的同构映射. \\
令$a \in A, \phi(a)=\bar{a} \in \bar{A}$. \\
由同构映射,有$\phi(\frac{a}{2}+\frac{a}{2})=\phi(\frac{a}{2})\cdot \phi(\frac{a}{2})$成立. \\
即$\phi(a)=\phi^2(\frac{a}{2})$ \\
由于$\bar{A}$是有理数集的子集,故$\phi^2(\frac{a}{2})>0$,即$\phi(a)>0$对任意$a \in A$成立. \\
意味着任意$\bar{b}<0$,且$\bar{b} \in \bar{A}$将找不到$\phi^{-1}$对应的$A$中原象,与$\phi$是一一映射矛盾,放$\phi$不存在.
\section*{\$1.10:等价关系与集合的分类}
1. $A=\{$所有实数$\}$. $A$的元间的关系$>$以及$\geq$是不是等价关系? \\
解:\\
对于关系$>$,显然不符合反射律,$ \forall a \in A, $不存在$ a>a $.故非等价关系
\\
对于关系$\geq$,符合反射律:$ \forall a \in A, $有$a \geq a$成立. 
不符合对称律:$\forall a,b \in A$且$a \neq b$,$a \geq b$与$b \geq a$只能成立其一.\\
故非等价关系.\\

2*. \\
解:\\
即便满足对称律与推移律.对于$a$而言,假如不存在一个使得$aRb$成立的$b$,就无法得到$aRa$.\\
反由对称律与推移律的存在无法确保如是的$b$存在.

3. 仿照例3规定整数间的关系:
\begin{equation*}
	a \equiv b (-5)
\end{equation*}
证明你所规定的是一个等价关系,并且找出模$-5$的剩余类.\\
证明: \\
反射律:$a \equiv a(-5)$显然成立. \\
对称律:$a \equiv b(-5) \Rightarrow b \equiv a(-5)$显然成立. \\
推移律:$a \equiv b(-5), b \equiv c(-5) \Rightarrow a \equiv c(-5)$,显然成应.\\
定义$-5$的剩余类$\{0,1,2,3,4\}$


\section*{\$2.6:置换群}
1.解:\\
$$
\left (\begin{array}{cccc}
	1 &2 & 3 \\
	1 &3 & 2  
\end{array}\right)
\left (\begin{array}{cccc}
	1 &2 & 3 \\
	2 &1 & 3  
\end{array}\right)
\left (\begin{array}{cccc}
	1 &2 & 3 \\
	3 &2 & 1  
\end{array}\right)
$$

2.解:\\
$$
\left (\begin{array}{cccc}
	1 &2 & 3 \\
	1 &2 & 3  
\end{array}\right)=(1) \ \ \ 
\left (\begin{array}{cccc}
	1 &2 & 3 \\
	1 &3 & 2  
\end{array}\right)=(2\  3) \ \ \ 
\left (\begin{array}{cccc}
	1 &2 & 3 \\
	2 &1 & 3  
\end{array}\right)=(1 \ 2)
$$
$$
\left (\begin{array}{cccc}
	1 &2 & 3 \\
	3 &2 & 1  
\end{array}\right)=(1 \ 3) \ \ \ 
\left (\begin{array}{cccc}
	1 &2 & 3 \\
	2 &3 & 1  
\end{array}\right)=(1\ 2 \ 3) \ \ \ 
\left (\begin{array}{cccc}
	1 &2 & 3 \\
	3 &1 & 2  
\end{array}\right)=(1 \ 3 \ 2)
$$
3.解:\\
(i)\\
对于不相挥的置换工,t最字Q如果出现在工工中 则只能出现在 2看之-,不妨设为工tb@am=(abbaut -lnj7_Oat=b
符全交换律;如果不出现在己t中,则飞对a均不发为生改变故 at.azz-a 符合交使津
(ii)\\
a;i-1)()(A@=a故(in-)(HH-)=E

4.解:\\


5.解:\\




\section*{\$2.7:循环群}
1.解:\\
设有循环群$G(a)$\\
$\forall a^{m_1},a^{m_2} \in G(a)$ \\
$a^{m_1}a^{m_2}=a^{m_1+m_2}=a^{m_2}a^{m_1}$\\
故有交换律.

2.解:\\
$a^n=e$,设$(a^r)^m=e$\\
$a^{rm}=e \ \ \ \ rm=kn$,$k$为正整数.\\
显然使$rm=kn$成立的最小$m=\frac{n}{d}$为$a^r$的阶.

3.解:\\
考虑$a^r$的阶即可,设为$m$,则$(a^r)^m=e$\\
已知$a^n=e$,故有$rm=kn$,$k$是正整数\\
由于$rm=o(r)$,故$kn \equiv o(r)$\\
若$(r,n)=1$,则$k \equiv o(r)$\\
故$m=\frac{kn}{r}$最小值为$n$,当$k=r$时成立.\\
即$a^r$的阶为$n$.其生成群与$n$的剩余类加群同构,即与$G$同构,显然$a^r$也生成$G$

4.解:\\
存在$G$到$\bar{G}$的变换$\phi$满射,设$G=(a)$.\\
$\phi: \ \ \ \ \ a \rightarrow \bar{a}$\\
对于$\bar{G}$中$\forall \bar{g}$,$\exists a^m \in G$使得$\phi(a^m)=\bar{g}$.\\
$\phi(a^m)=(\phi(a))^m=\bar{a}^m=\bar{g}$ \\
所以$\bar{G}=(\bar{a})$

5.解:\\
$G$与整数加群同构,整数加群与任意模$n$的剩余类加群同态\\
模$n$的剩余类加群与$n$阶循环群$\bar{G}$同构(若$\bar{G}$有限阶).故$G$与$\bar{G}$同态.\\
若$\bar{G}$无限阶,显然有$G$与$\bar{G}$同态(同与整数加群同构)

\end{document}