\documentclass{ctexart}
\usepackage{fancyhdr}
\usepackage{amsmath}
\usepackage{indentfirst}
\usepackage{graphicx}
\pagestyle{fancy}
\lfoot{}%这条语句可以让页码出现在下方
\title{《抽象代数》\\ 第五次作业} 
\author{\\姓名:姜岚曦 \ \ 学号:19375233
		\\姓名:魏来   \ \ 学号:20374104         
	    \\姓名:曹建钬 \ \ 学号:20375177
        \\姓名:李璞   \ \ 学号:20376164
        \\姓名:刘炅   \ \ 学号:21374261}
\date{}
\begin{document}
\pagestyle{empty}
\thispagestyle{empty}
\CTEXsetup[format={\Large\bfseries}]{section}
\maketitle
\clearpage	
	
\section*{\$2.8:子群}
1.
解:\\
$S_3=\{(1),(12),(23),(31),(123),(132)\}$\\
$H_1=\{(1)\} H_2=S_3$两平凡子群.\\
$H_3=\{(1),(12)\}$,$H_4=\{(1),(23)\}$,$H_5=\{(1),(31)\}$,$H_6=\{(1),(123),(132)\}$

2.
解:\\
设群$G$有两个子群$H_1$、$H_2$,有$e\in H_1,e \in H_2$.故$H_1 \bigcap H_2 \neq \emptyset $.\\
设$H_1$、$H_2$的交集中有一元素名为$a$,即$a \in H_1,a \in H_2$.\\ 
则有$a^{-1} \in H_1, a^{-1} \in H_2$,即$a^{-1} \in H_1 \bigcap H_2$\\
设另有一元素$b \in H_1 \bigcap H_2$. 则$ab \in H_1, ab \in H_2, b^{-1} \in H_1, b^{-1} \in H_2$ \\
故$ab \in H_1 \bigcap H_2, b^{-1} \in H_1 \bigcap H_2$\\
故$ H_1 \bigcap H_2$满足定理1,是$G$的子群

3.
解:\\
$S=\{(12),(13)\}$,设生成子群$H$ \\
则$(12)(123)=(13) \in H. \ \ \ \ (123)(12)=(23) \in H$\\
则$(12)(23)=(123) \in H. \ \ \ \ (12)(12)=(1)$\\
故$H=S_3$\\
肯定会,$S_3$也是$S_3$的一个子集,生成平凡子群与$S$生成的$H$一样

4.
解:\\
显然,循环群$G=(a)$的子群$\{e\}$也是循环群\\
设子群$H$中有元素$a^m$.令$i$是使$a^j \in H$的最小正整数. \\
则$H$中可写成$a^{iq+r}$的形式,其中$q$是正整数,$0 \leq r < i$\\
则$a^{iq} \in H \Rightarrow a^r \in H$,而因为$r<i$,故$r=0$ \\
$H=(a^i)$

5.
解:\\
$\{[0]\}$ \ \ $\{[0],[1],\ldots,[11]\}$ \\
$\{[0],[2],[4],[6],[8],[10]\}$\\
$\{[0],[3],[6],[9]\}$ \ \ $\{[0],[4],[8]\}$ \ \ $\{[0],[6]\}$

6.
解:\\
设$a$的阶是$m,a^m=e.$ \\
则$a^{m-1}=a^{-1} \in H,H$是子群.充分性得证.\\
必要性,由定理1,显然成立.

\section*{\$2.9:子群的陪集}
1.
解:\\
如果一个群$G$的阶是素数,设其为$N$.\\
由定理2有$N=nj$,其中$n$是$G$的一个子群$H$的阶,$j$是$H$在$G$里的指数.\\
由于$N$是素数,故$n=l$或$N$. \\
若$n=1$,则显然$H=\{e\}$,其$j=N$.即存在$N$个$H$的右陪集.\\
故有$H(a)=\{e,a\}$ 当$a \neq e, a\in G$时. \\
对于每一个符合条件的$a$,均有$\{a\}$是一个群,即有$a=e$成立矛盾.故$G$中只有一个元素$e$.矛盾.\\
故$n=N$,显然$H=G$,对于$G$中任意一个不为$e$的元$a$,$a$生成的循环子群的阶必为$N$.即有$G=(a)$.故$G$是循环群

2.
解:\\
设群$G$的阶为$P^m$. $P$是素数. $G$中任意不为$e$的元$a$,阶为$n$.\\
则由定理3有$n/p^m$. 故$n=p^t, t \geq 1$\\
当$t=1$时,$n=p$,$(a)$是一个阶为$p$的子群.\\
当$t>1$时,取$b=a^{p^{t-1}}$,有$b^p=a^{p^t}=a^n=e$\\
易证$b$的阶为$P$,$(b)$是一个阶为$P$的子群.

3.
解:\\
设群G的阶为N,则有$m/N,n/N$.\\
设$ab$的阶是P,则有$P/N$.\\
由$ab=ba$,则$(ab)^{mn}=a^{mn}b^{mn}=e$.\\
故$p/mn$.\\
考察$(ab)^{pn}=a^{pn}b^{pn}=a^{pn}=e$.\\
故$m/pn$,由于$(m,n)=1$,故$m/p$.\\
同理有$n/p$得$p=mn$.

4*.
解:
设H中两不为e的元素a,b,$a\in H,b \in H$,H是与e等价的元的集合.\\
故有$a\sim e,b\sim e。a\sim e \Rightarrow a\sim aa^{-1} \Rightarrow e\sim a^{-1} \Rightarrow a^{-1} \in H$.\\
$b\sim e \Rightarrow a^{-1}ab \sim a^{-1}a \Rightarrow ab\sim a\sim e \Rightarrow ab\sim e \Rightarrow ab\in H.$\\
故H是一个子群.

5.
解:\\
假设G中某元a同时属于2个右陪集$Hb,Hc$,$b \in G,c \in G$.\\
则$a=h_1b=h_2c$,其中$h_1 \in H, h_2 \in H$.\\
则有$b=(h_1)^{-1}h_2c$\\
对$Hb$中的任意元hb有$hb=h(h_1)^{-1}h_2c$,显然$h(h_1)^{-1}h_2 \in H$\\
故hb是$Hc$中的元,即$Hb \subset Hc$,同理可有$Hc \subset Hb$.\\
即$Hb=Hc$矛盾,故a至多属于一个右陪集. \\
显然由于H是子群,势必包含e,则$a=ea \in Ha$.\\
故a属于1个右陪集.

6*.
解:\\
对于一个阶为4的群G而言,其中元的阶只能是l,2.4\\
若G有一个阶为4的元a,则(a)是一个4阶循环群与模4剩余类加群同构.\\
若G无阶为4的元,则G有3个阶为2的元,设为$xyz$.\\
则$G=\{e,x,y,z\}$.显然$xy \in G$.\\
若$xy=e$.则$xy^2=y=x$,矛盾\\
若$xy=x$,则$y=e$,矛盾;若$xy=y$,则$x=e$,矛盾\\
所以$z=xy$,同理有$xy=yx=z \ \ \ \ yz=zy=x \ \ \ xz=zx=y$\\
上述所有群都与$G=\{e,x,y,z\}$对于运算R的群同构,其中:
\begin{center}
	\begin{tabular}{c|cccc}
		R & e & x & y & z\\
		\hline
		e & e & x & y & z\\
		x & x & e & z & y\\
		y & y & z & e & x\\
		z & z & y & x & e\\		
	\end{tabular} 
\end{center}
综上,对于任何一个4阶群,其必与$G$或模4的剩余类加群二者之一同构.
\section*{\$2.10:不变子群. 商群}
1.
解:\\
不变子群$N$阶为2,设其为$\{e,a\}$\\
其中有$a=a^{-1},a^2=e$成立,由定理2.\\
对于$\forall x \in G,$有$xax^{-1} \in \{e,a\}=N$.\\
若$xax^{-1}=e$,则$xa=x \Rightarrow a=e$不可能\\
故$xax^{-1}=a$,即$xa=ax$对$\forall x \in G$成立.\\
显然$G$的中心包含$a,e$.显然故包含$N$

2.
解:\\
首先,两个子群的交集一定还是一个子群\\
令两个子群的交集$N=N_1 \bigcap N_2$\\
$N$中一元设为$n$,则有$n \in N_1, n \in N_2$\\
由定理2,有$\forall a \in G,$有$ana^{-1} \in N_1,ana^{-1} \in N_2$\\
即$ana^{-1} \in N_1 \bigcap N_2 = N \Rightarrow ana^{-1} \in N$对$\forall a \in G$成立.\\
故$N=N_1 \bigcap N_2$是不变子群

3.
解:\\
设N是G的一个指数为2的子群,显然$N \neq G.$\\
对于G中两元e、b,其中b是任意不为e的元,且$b \bar{\in} N.$\\
N的两个右陪集为$Ne=N$与$Nb$,两者构成G的一个分类.\\
同理有两个左陪集$eN=N$与$bN$亦构成C的一个分类.\\
所以有$Nb=bN$对任意符合要求的b成立\\
而对于$a\in N$而言,$aN=Na$显然成立,故N是不变子群

4.
解:\\
设HN中元为hn,其中$h \in H, n \in N$.\\
设HN中两元$h_1n_1,h_2n_2,$其中$h_1,h_2 \in H, n_1,n_2 \in N$\\
则$(h_1n_1)(h_2n_2)^{-1}=(h_1n_1)(n_2^{-1}h_2^{-1})=h_1n_1n_2^{-1}h_2^{-1}=h_1(n_1n_2^{-1})h_2^{-1}=h_1nh_2^{-1}$\\
显然$n \in N$,所以$nh_2^{-1}$构成的集合同$h_2^{-1}n$\\
故$h_1nh_2^{-1} \in h_1HN=HN$.\\
故HN是G的子群.

5.
解:\\
$G=S_4,N=\{(1),(12)(34),(13)(24),(14)(23)\}$\\
$N_1=\{(1)(12)(34)\}$\\
$N_1$是$N$的一个不变子群(由于$N$是交换群)\\
但容易找到$ana_1 \bar{\in} N_1$的例子,其中$a \in G, n \in N_1$

6*.
解:\\
(i) C显然是个子群\\
对$\forall a \in G,c \in C, aca^{-1}=(aca^{-1}c^{-1})c \in C$\\
故C是一个不变子群\\
(ii)令$a,b \in G$,则有$c=a^{-1}b^{-1}ab \in C$\\
故有$ab=bae \ \ \ \ \ abcC=bacC=baC$\\
故有$CabC = CbaC=aCbC=bCaC$\\
故$a/c$是交换群\\
(iii)对$\forall a,b \in G$,有$aNbN=bNaN$\\
即$abNN=baNN \Rightarrow abN=baN$\\
即有$abe=baN$, $n \in N \Rightarrow ab=ban \Rightarrow a^{-1}b^{-1}ab=n \in N$包含C\\
故$C \subset N$.

\section*{\$2.11:同态和不变子群}
1. 
解:\\
由于$\phi$是满射,则$\forall \bar{a} \in \bar{S}$,存在$a \in S$使$\phi(a)=\bar{a}$,有$\bar{S} \subset \phi(S)$. 对于$\forall a \in S$,存在$\bar{a} \in \bar{S}$使$\phi(a)=\bar{a}$,有$\phi(S) \subset \bar{S}$,故$\phi(S)=\bar{S}$\\
反例很容易举出,设$A=\{a,b,c\}$,$\bar{A}=\{c\}$.$\phi: \ \ \ \ \ a\rightarrow c,b \rightarrow c, c \rightarrow c$. \\
取$S=\{a\}$,则$\bar{S}=\{c\}$.而$\bar{S}$的逆像是$\{a,b,c\} \neq {a} = S$

2. 
解:\\
跳过

3. 
解:\\
$G$与$\bar{G}$同态$\Rightarrow G/N \cong \bar{G}$. $G/N$的阶为$n$是$N$在$G$的指数显然$n/m$.\\
若$n/m$,令$G=(a), \bar{G}=(\bar{a})$\\
$\phi: \ \ \ \ a^k \rightarrow \bar{a}^k$
若$a^{k_1}=a^{k_2}$,则$m/k_2-k_1 \Rightarrow n/k_2-k_1 \Rightarrow \bar{a}^{k_1}=\bar{a}^{k_2}$故是映射.\\
显然对$\forall \bar{k} \in \bar{G}$都能找到$\phi^{-1}(\bar{a}^k)=a^k \in G$.\\
故$\phi$是同态满射,$G$与$G$同态.

4.
解:\\
循环群是交换群,故其子群是不变子群,$G/N$存在.\\
设$G=(a)$,对于$\forall a^k \in G$,存在$N$的陪集$a^kN=(aN)^k$\\
全体$a^kN$构成$G/N$亦即$(aN)$是循环群.
\end{document}
